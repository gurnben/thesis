%
% results_server.tex
%

\subsection{Adding a Dataset}

One of the most common avenues for change in any data visualization application is the data in use.  In our architecture, datasets are acquired from different data sources/paths through datapaths.  In order to provide a new dataset, the developer should verify that their datapaths have access to this dataset.  If all datapaths simply query dynamically from a database based on request parameters and that database contains the new dataset, the datapaths should have access to the new dataset.  If queries are not dynamically built, then the developer will have to add the new dataset to all datapaths that don’t include it.  By writing dynamic datapaths, some work can be avoided when adding new datasets.  Once all datapaths have access to the new dataset, the developer has to add that dataset to the allowed datasets list in their configuration file.  As a general security measure, datasets have to be whitelisted in order to be included in visualizations, which is addressed in this final step.  After this step, the new dataset should be available to the client and can be requested by a chart.  So, in conclusion, in order to add a new dataset, the developer needs to validate that the dataset is available to all datapaths and that it is present in the allowed datasets list. \par

\subsection{Adding a Datapath}
Another common avenue of change lies in the datapaths themselves.  Each datapath defines a data scale and data aggregation method through which the available datasets can be viewed.  A common reason to add a datapath would be to accommodate different data focus levels.  For example, if data is scaled by time, an application may include a data scale of 1 point per day and 1 point per hour.  If the developer finds that users often query a datetime range such that aggregating the points bi-hourly would be more appropriate, they can add a datapath for that data scale. In this case, the developer need only define a new datapath that queries and aggregates data at a new scale and place it in the datapaths folder.  When the server is restarted, the new datapath will be dynamically loaded and utilized when appropriate. \par

\subsection{New Data Channels/Sources}
In our architecture, different data sources are loosely encapsulated within data channels.  These data channels, as discussed in section 5.2.2, represent a single source of data that could be a database collection/table, a web API hook, or any real source of data.  Data channels should provide some unified interface.  It is highly likely that during the lifetime of a data visualization web application, the developer may want to add a new table/collection, change database solutions, or have to adapt to a change in a web API.  As such, data channels are encapsulated within their respective datapaths and can be easily added or updated.  In order to add a new data channel, the developer must first revise datapaths.  Each datapath that needs to have access to the new data channel should be revised to include code to query the new data channel.  Keep in mind that the data channel accessing code can be defined once and reused by all data paths as desired.  Once complete, they must add the new data channel’s available data sets to the allowed datasets list in their configuration file.  Once the server is restarted, the new data channel and all contained datasets will be available to the client. \par
