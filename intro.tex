%
% intro.tex
%

The Internet’s foundational technologies are perpetually evolving.  As the underlying languages and technologies change, our methods to solve common problems must also adapt to leverage new capabilities, overcome newfound weaknesses, and offer experiences possible only with advancing technology. Over the past decade we have observed a series of dramatic innovations in web technology, and as such we have observed a paradigm shift as mostly static websites transformed into dynamic web applications.  Web applications, although hard to define, typically focus on providing responsive experiences that incorporate more complex and dynamic functionality.  Web applications represent the next logical progression for both websites and desktop applications.  Websites see a growing demand for dynamic and increasingly complex functionality leading to their conversion to web applications.  Desktop applications target multiple platforms, integrate live web services, and aim to feature cloud data synchronization leading to their natural evolution into web applications.  As a concrete example of this paradigm shift, observe the evolution of document editing software.  Document creation and editing software such as Microsoft Word and Latex originated in the 1980s to fulfill professional and personal document-focused tasks.  These applications leveraged the newfound power of the personal computer.  These were standalone, static pieces of software that provided complex document creation and editing functionality.  Today, applications such as Google Docs, Microsoft Word Online, and LaTeX editing applications such as OverLeaf serve as prime examples of traditional applications evolved into web applications. These web applications offer the same functionality as their traditional desktop counterparts; however, they exist as entirely cloud-based browser-driven applications.  These new solutions also extend their more traditional application counterparts by including the ability to collaborate in real-time, automatically save to a remote cloud server, and easily import web-based plugins for new features.  Both websites and desktop applications can effectively use the power of modern web applications to their advantage. \par
As mentioned previously, the rise of web applications was accompanied by a revolution in web technologies.  HTML (Hypertext Markup Language), a markup language used to define the elements on a web page, and CSS (Cascading Style Sheets), a style sheet language used to define styling on rendered HTML elements, play a foundational role in any web-based project.  As web technologies evolved, HTML and CSS maintained their prominent role as the pillars of web development, becoming an accepted standard for defining and styling content to be rendered in a web browser.  As the Internet continued to grow in size and utility, the demand for dynamic behavior, data rendering, and interaction continued to grow.  Various technologies arose to tackle this issue, notably including PHP (Hypertext Preprocessor) for pre-processed HTML, and other solutions such as Java Applets, Adobe Flash Player, and finally JavaScript.  JavaScript presented itself as a high-level interpreted programming language with broad capabilities.  It also touted that it could be natively integrated into a web site/application and executed in-browser.  JavaScript slowly claimed a foundational role alongside HTML and CSS as it was used to add dynamic functionality to a web site.  Due to its simple integration inside HTML files, it could easily be added to standard HTML/CSS websites or even PHP preprocessed sites.  With further technological advancements including the development of the V8 JavaScript engine, JavaScript became a viable, useful, and sufficiently efficient web programming language. During this time, server-side technologies followed a similar path, evolving from simple file serving, to PHP-driven preprocessed sites using Apache as a typical web server, and eventually to JavaScript driven web servers using Node.js built upon the V8 JavaScript engine.  Now, JavaScript serves as a viable language for both server and client side web development, offering a unified language and knowledge set. \par
Alongside the shift from traditional applications and websites to web applications, the industry grew to use technology in a more data-centric role.  Logically, this led to a shift towards web-based data visualizations for commercial, scientific, and personal use.  Web applications grew to offer professional-grade functionality in the form of various “dashboards” for different products.  Take, for example, GitHub’s current web presence, making use of various dynamically-populated visualizations to show different contributions to a project over time.  These data-driven web applications have flooded the commercial and non-commercial scene, with commercial examples such as GitHub, research projects such as BioJS, and various non-commercial endeavours.  Although web-based data visualization products, platforms, or open source projects vary in their targeted audience, they all aim to solve a common data visualization problem.  Unfortunately, no single approach to web based data visualization arose as a clear standard, leaving the space fragmented. \par
An ideal solution to the web-based data visualization problem would have to be adaptable, flexible, charting platform agnostic, and leverage new full-stack JavaScript capabilities while remaining platform and framework agnostic.  It is critical that any architecture that aims to solve the data visualization problem on the web must adapt to the type of data to be visualized, the purpose of the visualization, the methods used to render a chart, and any common web frameworks/platforms on the client and server side.  Any ideal solution should allow, for example, the visualization of anything from simple numerical data, such as a daily high temperature, to complex 3-dimensional model visualization, such as the visualization of an MRI scan.  In order to achieve these goals, any general-purpose architecture should define high-level structural elements while keeping a careful mind to their application in various common frameworks.  It would also be beneficial to develop a use-case scenario and implement any proposed architecture in a modern context to illustrate its form and function.  Any proof of concept should also have a specific focus on its ability to leverage modern web technologies.  This thesis outlines and applies such an architecture. \par
Our architecture for web-based data visualization focuses on adaptability, flexibility, and platform agnosticism by defining a high-level structure.  Although different platforms will have different implementations, this structure prescribes certain abstractions, inheritance structures, and modular components that can be applied on different platforms and frameworks affectively.  For example, in the common client-side framework, Angular, charting components can be defined quite literally as Angular components that implement a chart component interface.  The same application in another client-side framework, React, would include a series of “components” separated into different files that all implemented a common set of functions.  These files would follow an inheritance structure, but without any language-level enforcement unless TypeScript were used.  These files would be imported by the primary page.  With this in mind, our architecture was designed structurally and then applied to a web application built using the MEAN (MongoDB, ExpressJS, Angular, Node.js) web stack.  Our web application, called BeeStream, is tasked with streaming honey bee hive videos from embedded hive monitoring systems under the Beemon project.  As part of this thesis, Beestream was extended to include visualizations for quantitative analyses of automatically ingested and analyzed honey bee hive video and audio data.  This application allowed us to further refine our architecture with modularity and adaptability in mind, and it serves as a practical example of the architecture’s effectiveness. \par
