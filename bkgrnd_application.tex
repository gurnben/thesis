%
% bkgrnd_application.tex
%

	Our architecture was applied as part of the Beestream web application.  In order to understand the motivations for this project, the reader should have a general understanding of the Beemon project.  Beemon is a honey bee hive monitoring application that records video, audio, and temperature/humidity data at the entrance of a honeybee hive.  The recorded data is uploaded to a remote server where it can be accessed and analyzed.  As part of the Beemon project, a number of automated video and audio analytics programs were created.  These applications produce quantitative data, such as beehive arrivals and departures, from qualitative video and audio data.  When video or audio data is uploaded to the server, it is ingested into a MongoDB database and analyzed using these analytics programs.  The analysis results are included in the MongoDB database.  Beestream’s original purpose was to allow users to stream the video recorded by Beemon through a web application and comment/tag videos with interesting observations.  Given the readily available analysis, the idea was presented that Beestream could include various visualizations of the live honey bee hive data such as arrivals, departures, and temperature/humidity.  Our architecture was developed and applied to this visualization problem. \par
